\documentclass[12pt]{article}

\usepackage[T2A]{fontenc}
\usepackage[english,russian]{babel}
\usepackage[utf8]{inputenc}
\usepackage{indentfirst}
\usepackage[left=1.5cm, right=1.5cm, top=1.5cm, bottom=2cm]{geometry}

\begin{document}

\begin{flushright}
    Гусарова Мария, 151 гр. КНиИТ
\end{flushright}

\begin{center}
    \textbf{\Large{Глоссарий}}
\end{center}

\begin{enumerate}
    \item \textbf{Agile Methodology} (Никита Барабанов) "--- итеративная модель разработки, в которой программное обеспечение создают инкрементально с самого начала проекта, в отличии от каскадных моделей, где код доставляется в конце рабочего цикла.
    
    \item \textbf{API (Application Programming Interface)} (Никита Барабанов) "--- описание способов взаимодействия одной компьютерной программы с другими. Обычно входит в описание какого"=либо интернет"=протокола, программного каркаса (фреймворка) или стандарта вызовов функций операционной системы.

    \item \textbf{Backend} (Никита Барабанов) "--- программно"=аппаратная часть сервиса, отвечающая за функционирование его внутренней части.

    \item \textbf{Backend"=тестирование} "--- метод тестирования, который проверяет базу данных или серверную часть веб"=приложения. Основная цель внутреннего тестирования "--- проверка уровня приложения и уровня базы данных. Он найдет ошибку или ошибку в базе данных или на стороне сервера.

    \item \textbf{Backlog} (Никита Барабанов) "--- упорядоченный и постоянно обновляемый список всего, что планируется сделать для создания и улучшения продукта.

    \item \textbf{Big data} (с англ. <<большие данные>>) (Алексей Кузьмин) "--- это структурированные или неструктурированные массивы данных большого объема. Их обрабатывают при помощи специальных автоматизированных инструментов, чтобы использовать для статистики, анализа, прогнозов и принятия решений.

    \item \textbf{CTOR (Click"=To"=Open Rate)} (Павел Пасеков) "--- метрика в email маркетинге, которая показывает процентное соотношение кликов по ссылке в письме к количеству открытий.

    \item \textbf{Data mining} (Алексей Кузьмин) "--- собирательное название, используемое для обозначения совокупности методов обнаружения в данных ранее неизвестных, нетривиальных, практически полезных и доступных интерпретации знаний, необходимых для принятия решений в различных сферах человеческой деятельности.

    \item \textbf{Data science} (Алексей Кузьмин) "--- наука о данных, объединяющая разные области знаний: информатику, математику и системный анализ. Сюда входят
    методы обработки больших данных (Big Data), интеллектуального анализа данных (Data Mining), статистические методы, методы искусственного интеллекта, в т. ч. машинное обучение (Machine Learning).

    \item \textbf{Deep learning} (Алексей Кузьмин) "---  совокупность методов машинного обучения, основанных на обучении представлениям, а не специализированных алгоритмах под конкретные задачи.

    \item \textbf{DevOps} (акр. от <<development \& operations>>) (Алексей Кузьмин) "--- методология автоматизации технологических процессов сборки, настройки и развертывания программного обеспечения. Методология предполагает активное взаимодействие специалистов по разработке со специалистами по информационно"=технологическому обслуживанию и взаимную интеграцию их технологических процессов друг в друга для обеспечения высокого качества
    программного продукта.

    \item \textbf{Fast data} (Алексей Кузьмин) "--- совокупность технологий, обеспечивающих скоростную обработку потоков данных, генерируемых различными системами и активностями, их анализ и принятие решений с учетом исторически накопленных данных, а также автоматическое выполнение различных действий в ответ на произошедшие события.

    \item \textbf{Frontend} (Никита Барабанов) "--- клиентская сторона пользовательского интерфейса к программно-аппаратной части сервиса.
    
    \item \textbf{GAN (Generative Adversarial Network)} (Алексей Кузьмин) "--- алгоритм машинного обучения, входящий в семейство порождающих моделей и построенный на комбинации из двух нейронных сетей: генеративная модель, которая строит приближение распределения данных, и дискриминативная модель, оценивающая вероятность, что образец пришел из тренировочных данных, а не сгенерированных моделью.

    \item \textbf{GitLab} (Никита Барабанов) "--- инструмент для хранения и управления репозиториями Git. Он дает возможность выполнять совместную разработку силами нескольких команд, применять обновления кода и откатывать изменения, если это необходимо.

    \item \textbf{GNU Autotools} (Надежда Демина) "--- система сборки проекта GNU, набор программных средств, предназначенных для поддержки переносимости исходного кода программ между UNIX"=подобными системами.

    \item \textbf{JTAG (Joint Test Action Group)} (Никита, КРЭТ КБПА) "--- промышленный стандарт для проверки конструкций и тестирования печатных плат после изготовления.
    
    \item \textbf{Merge} (Надежда Демина) "--- внесение изменений в своей версии по сравнению с версией из репозитория.

    \item \textbf{MLOps} (Алексей Кузьмин) "--- набор процедур, направленных на последовательное и эффективное внедрение и поддержку моделей машинного обучения (ML), используемых в производстве. Само слово представляет собой сочетание, обозначающее <<Machine Learning>> (Машинное обучение) и процесс непрерывной разработки <<DevOps>> в области программного обеспечения.

    \item \textbf{Monolithic architecture} (Никита Барабанов) "--- традиционная модель программного обеспечения, которая представляет собой единый модуль, работающий автономно и независимо от других приложений. Монолитом часто называют нечто большое и неповоротливое, и эти два слова хорошо описывают монолитную архитектуру для проектирования ПО. 

    \item \textbf{Neoflex Reporting (NR)} (Алексей Кузьмин) "--- система аналитики и подготовки отчетности.

    \item \textbf{Open"=source} (Никита Рыданов) "--- программное обеспечение, которое поставляется для конечного пользователя с открытым исходным кодом. То есть приложение можно доработать под свои задачи без нарушения авторских прав разработчиков исходного ПО. 

    \item \textbf{Pipeline} (с англ. <<трубопровод>>) (Никита Барабанов) "--- это документ, визуализирующий процесс разработки продукта. Он представляет собой последовательность этапов, расположенных так, что конец предыдущего является началом следующего.

    \item \textbf{Proxy} (Игорь Юрин) "--- сервер"=посредник между пользователем и интернет"=ресурсом. Человек подключается не напрямую к серверу нужного сайта, а к прокси"=серверу — и уже он передает данные на сайт и отправляет обратно в браузер пользователя.

    \item \textbf{QA"=инженер} (акр. от <<Quality Assurance Engineer>>) (Павел Пасеков) "--- специалист, задача которого заключается в контроле за правильностью выполнения всех этапов разработки и правильностью работы итогового продукта.

    \item \textbf{RESTful API} (Никита Барабанов) "--- интерфейс,используемый двумя компьютерными системами для безопасного обмена информацией через Интернет. 

    \item \textbf{Roadmap} (с англ. дорожная карта)  (Павел Пасеков) "--- графический обзор целей и результатов проекта, представленных на временной шкале. В отличие от плана проекта, где детали обрисованы, дорожная карта должна быть простой и не содержать мелочи.

    \item \textbf{SOA (Service"=Oriented Architecture)} (Никита Барабанов) "--- метод разработки программного обеспечения, который использует программные компоненты, называемые сервисами, для создания бизнес"=приложений.
    
    \item \textbf{Soft skills} (с англ. <<мягкие навыки>>) (Павел Пасеков) "--- универсальные социально"=психологические качества, которые не зависят от профессии, но непосредственно влияют на успешность человека. К ним относятся коммуникативные навыки, организованность, способность решать конфликты, умение убеждать, работать в команде, адаптивность.
    
    \item \textbf{SRE (Site Reliability Engineering)} (Павел Пасеков) "--- сфера обеспечения бесперебойной работы высоконагруженных сервисов.
  
    \item \textbf{TeamCity} (Надежда Демина) "--- серверное программное обеспечение, написанное на языке Java, разработанное компанией JetBrains, которое обеспечивает непрерывную интеграцию (CI) кода.

    \item \textbf{UML (Unified Modeling Language)} (Максим, КРЭТ КБПА) "--- унифицированный язык моделирования. Это графический язык, который с помощью диаграмм и схем описывает разнообразные процессы и структуры.

    \item \textbf{UX/UI (User Experience)}  (Алексей Кузьмин) "--- то, каким образом пользователь взаимодействует с интерфейсом и насколько сайт или приложение для него удобны.

    \item \textbf{Waterfall Model} (с англ. <<каскадная модель>>) (Никита Барабанов) "--- водопадная или каскадная разработка продуктов. Она подобно потоку воды направляет команды решать задачи последовательно и строго по изначальному плану. 

    \item \textbf{Web"=тестирование} (Павел Пасеков) "--- процесс проверки клиент"=серверных продуктов, размещаемых в доступе посредством сети, зачастую включающий поддержку множества взаимозависимых элементов.

    \item \textbf{Аппроксимация} (Алексей Кузьмин) "--- определение параметров аналитической функции, описывающей набор точек, полученных в результате эксперимента.

    \item \textbf{Архитектура трансформера} (Никита Рыданов) "--- относительно новый тип нейросетей, направленный на решение последовательностей с легкой обработкой дальнодействующих зависимостей. На сегодня это самая продвинутая техника в области обработки естественной речи (NLP).

    \item \textbf{Аутсорсинговая компания} (Павел Пасеков) "--- фирма, которая предоставляет услуги высококвалифицированных специалистов для выполнения работ на постоянной основе.

    \item \textbf{Баг} (англ. <<bug>>, сленг) (Никита Барабанов) "---  жаргонное называние программной ошибки.

    \item \textbf{Бортовое ПО} (Никита, КРЭТ КБПА) "--- программное обеспечение бортовой ЭВМ встраиваемой системы.

    \item \textbf{Бюджет ошибок} (Павел Пасеков) "--- максимальное время, в течение которого техническая система может выходить из строя без оговоренных в соглашении последствий.

    \item \textbf{Деплой} (Никита Барабанов) "--- это развертывание и запуск веб"=приложения или сайта в его рабочей среде, то есть на сервере или хостинге. Разработчик загружает приложение, написанное на локальном компьютере, в специальное пространство, из которого оно доступно в интернете.

    \item \textbf{Кластеризация} (Надежда Демина) "--- разбиение множества объектов на подмножества (кластеры) по заданному критерию. Каждый кластер включает максимально схожие между собой объекты.

    \item \textbf{Компьютерная безопасность} (Игорь Юрин) "--- меры безопасности, применяемые для защиты вычислительных устройств (компьютеры, смартфоны и другие), а также компьютерных сетей (частных и публичных сетей, включая Интернет).

    \item \textbf{Конечный автомат} (Леонид, ЦОПП) "--- математическая абстракция, модель дискретного устройства, имеющего один вход, один выход и в каждый момент времени находящегося в одном состоянии из множества возможных.

    \item \textbf{Лог} (Никита Барабанов) "--- текстовый файл, куда автоматически записывается важная информация о работе системы или программы. Чаще всего говорят о логах сервера. Их записывает программное обеспечение, которое управляет внутренней частью сайта или онлайн"=системы.

    \item \textbf{Микросервисы} (Никита Барабанов) "--- способ организации приложения с помощью нескольких независимых однофункциональных протоколов или сервисов.

    \item \textbf{Многослойный перцептрон} (Алексей Кузьмин) "--- класс искусственных нейронных сетей прямого распространения, состоящих как минимум из трех слоев: входного, скрытого и выходного. За исключением входных, все нейроны использует нелинейную функцию активации.

    \item \textbf{Модуль MVC} (с англ. <<модель"=представление"=контроллер>>) (Иван Жадаев) "--- способ организации кода, который предполагает выделение блоков, отвечающих за решение разных задач. Один блок отвечает за данные приложения, другой отвечает за внешний вид, а третий контролирует работу приложения.

    \item \textbf{Паттерны программирования} (Иван Жадаев) "--- способы построения программ, которые считаются хорошим тоном для разработчиков. Их еще называют шаблонами или образцами: чаще всего паттерн "--- это типовое решение для часто встречающейся задачи на построение.

    \item \textbf{Плюшки} (сленг) (Павел Пасеков) "--- специальные бонусы в работе программиста.

    \item \textbf{Продакшн} (англ. <<production>>) (Никита Барабанов) "--- запущенная версия сайта, та, которую видят пользователи.

    \item \textbf{Профит} (англ. <<profit>>, сленг) (Никита Барабанов) "--- выгода, прибыль.

    \item \textbf{Пушить (git)} (Никита Барабанов) "--- сленг, образованный от консольной команды <<git push>>, которая передает в удаленный репозиторий изменения, сделанные в локальном репозитории. С помощью этой консольной команды разработчики дорабатывают основную ветку, добавляя новые фичи и внося исправления найденных багов и уязвимостей.

    \item \textbf{Режим dev"=test"=prod} (с англ. <<разработка"=тестирование"=эксплуатация>>) (Никита Барабанов) "--- режим, который предполагает, что разработка и тестирование новых процессов ведутся не в рабочем приложении (где работают обычные бизнес"=пользователи), а в отдельных приложениях. В рабочее приложение переносятся только проверенные, протестированные процессы. Режим dev"=test"=prod увеличивает стабильность системы и удобство разработки.
    
    \item \textbf{Рекуррентные нейронные сети} (англ. <<recurrent neural network>>) (Алексей Кузьмин) "--- вид нейронных сетей, где связи между элементами образуют направленную последовательность. Благодаря этому появляется возможность обрабатывать серии событий во времени или последовательные пространственные цепочки.

    \item \textbf{Сверточные нейронные сети}  (англ. <<convolutional neural network>>) (Алексей Кузьмин) "--- специальная архитектура искусственных нейронных сетей, предложенная Яном Лекуном в 1988 году и нацеленная на эффективное распознавание образов, входит в состав технологий глубокого обучения.

    \item \textbf{Толстый/тонкий клиент} (Иван Жадаев) "--- условное разделение всех существующих клиентов. Толстый клиент — это клиент, обеспечивающий полную функциональность и независимость приложения от центрального сервера. Тонкий клиент не выполняет никаких задач, связанных с обработкой данных. Вместо этого все вычислительные мощности переносятся на удаленный сервер, с которым он взаимодействует посредством терминального доступа.

    \item \textbf{Точка доступа} (Никита, КРЭТ КБПА) "--- базовая станция, предназначенная для обеспечения беспроводного доступа к уже существующей сети (беспроводной или проводной) или создания новой беспроводной сети.

    \item \textbf{Трассировка} (Максим, КРЭТ КБПА) "--- процесс пошагового выполнения программы. В режиме трассировки программист видит последовательность выполнения команд и значения переменных на данном шаге выполнения программы, что позволяет легче обнаруживать ошибки.

    \item \textbf{Троян} (англ. <<trojan>>, акр. от <<троянская вирусная программа>>) (Игорь Юрин) "--- разновидность вредоносной программы, проникающая в компьютер под видом легитимного программного обеспечения, в отличие от вирусов и червей, которые распространяются самопроизвольно.

    \item \textbf{Фреймворк} (с англ. <<остов, каркас, структура>>) (Алексей Кузьмин) "---  программная платформа, определяющая структуру программной системы; программное обеспечение, облегчающее разработку и объединение разных компонентов большого программного проекта.

    \item \textbf{Черный ящик} (Максим, КРЭТ КБПА) "--- термин, используемый для обозначения системы, внутреннее устройство и механизм работы которой очень сложны, неизвестны или неважны в рамках данной задачи. 

    \item \textbf{Юзабельный} (англ. <<usable>>, сленг) (Алексей Кузьмин) "--- пригодный для использования, используемый.

    \item \textbf{Язык запросов} (Иван Жадаев) "--- искусственный язык, на котором делаются запросы к базам данных и информационно"=поисковым системам.
\end{enumerate}

\end{document}