\documentclass[referat]{SCWorks}
% Тип обучения (одно из значений):
%    bachelor   - бакалавриат (по умолчанию)
%    spec       - специальность
%    master     - магистратура
% Форма обучения (одно из значений):
%    och        - очное (по умолчанию)
%    zaoch      - заочное
% Тип работы (одно из значений):
%    coursework - курсовая работа (по умолчанию)
%    referat    - реферат
%  * otchet     - универсальный отчет
%  * nirjournal - журнал НИР
%  * digital    - итоговая работа для цифровой кафдры
%    diploma    - дипломная работа
%    pract      - отчет о научно-исследовательской работе
%    autoref    - автореферат выпускной работы
%    assignment - задание на выпускную квалификационную работу
%    review     - отзыв руководителя
%    critique   - рецензия на выпускную работу
% Включение шрифта
%    times      - включение шрифта Times New Roman (если установлен)
%                 по умолчанию выключен


\usepackage{preamble}
%\renewcommand{\baselinestretch}{1.5}
\linespread{1.5}
\setlength{\parindent}{1.25cm}




\makeatletter
%\renewcommand{\@oddfoot}{\hfil \small \arabic{page}\hfil}
%\renewcommand{\@evenfoot}{\hfil \small \arabic{page}\hfil}

\renewcommand{\@oddfoot}{\small \vbox{\hbox to\textwidth{\hfil\thepage}}}
\renewcommand{\@evenfoot}{\small \vbox{\hbox to\textwidth{\thepage\hfil}}}
\makeatother



%\usepackage{fancyhdr}
%\pagestyle{fancy}
%\fancyhf{}
%\fancyfoot[R]{\thepage}
%\renewcommand{\headrulewidth}{0pt}
%\renewcommand{\footrulewidth}{0pt}

\begin{document}

% Кафедра (в родительном падеже)
\chair{информатики и программирования}

% Тема работы
\title{КОМПЬЮТЕРНАЯ ГРАФИКА}

% Курс
\course{1}

% Группа
\group{151}

% Факультет (в родительном падеже) (по умолчанию "факультета КНиИТ")
\department{факультета компьютерных наук и информационных технологий}

% Специальность/направление код - наименование
% \napravlenie{02.03.02 "--- Фундаментальная информатика и информационные технологии}
% \napravlenie{02.03.01 "--- Математическое обеспечение и администрирование информационных систем}
% \napravlenie{09.03.01 "--- Информатика и вычислительная техника}
\napravlenie{09.03.04 Программная инженерия}
% \napravlenie{10.05.01 "--- Компьютерная безопасность}

% Для студентки. Для работы студента следующая команда не нужна.
\studenttitle{студентки}

% Фамилия, имя, отчество в родительном падеже
\author{Гусаровой Марии Владимировны}

% Заведующий кафедрой 
% \chtitle{доцент, к.\,ф.-м.\,н.}
% \chname{С.\,В.\,Миронов}

% Руководитель ДПП ПП для цифровой кафедры (перекрывает заведующего кафедры)
% \chpretitle{
%     заведующий кафедрой математических основ информатики и олимпиадного\\
%     программирования на базе МАОУ <<Ф"=Т лицей №1>>
% }
% \chtitle{г. Саратов, к.\,ф.-м.\,н., доцент}
% \chname{Кондратова\, Ю.\,Н.}

% Научный руководитель (для реферата преподаватель проверяющий работу)
%\satitle{доцент, к.\,ф.-м.\,н.} %должность, степень, звание
%\saname{Г.\,Г.\,Наркайтис}

% Руководитель практики от организации (руководитель для цифровой кафедры)
%\patitle{доцент, к.\,ф.-м.\,н.}
%\paname{С.\,В.\,Миронов}

% Руководитель НИР
%\nirtitle{доцент, к.\,п.\,н.} % степень, звание
%\nirname{В.\,А.\,Векслер}


% Год выполнения отчета
\date{2023}

\maketitle

% Включение нумерации рисунков, формул и таблиц по разделам (по умолчанию -
% нумерация сквозная) (допускается оба вида нумерации)
\secNumbering

\tableofcontents

% Раздел "Обозначения и сокращения". Может отсутствовать в работе
% \abbreviations
% \begin{description}
%     \item ... "--- ...
%     \item ... "--- ...
% \end{description}

% Раздел "Определения". Может отсутствовать в работе
% \definitions

% Раздел "Определения, обозначения и сокращения". Может отсутствовать в работе.
% Если присутствует, то заменяет собой разделы "Обозначения и сокращения" и
% "Определения"
% \defabbr

\intro
Компьютерная графика (КГ) представляет собой отдельную область человеческой деятельности, в рамках которой выполняется создание и обработка изображений посредством компьютеров и специального ПО. Кроме того, под компьютерной графикой подразумевают метод трансформации реальной визуальной информации в оцифрованную форму, а также ее сохранение и изменение.

Компьютерная графика "--- это результат внедрения в искусство новейших технологий обработки данных, позволяющих художнику без использования традиционных инструментов и материалов решать важные творческие задачи:

\begin{itemize}
    \item создавать всевозможные визуальные и анимационные эффекты; 
    \item изменять цвет и форму любого объекта;
    \item создавать художественные образы с помощью линий, штриховки и пятен.
\end{itemize}

Первоначально программисты научились получать рисунки в режиме символьной печати. На бумажных листах с помощью символов (звездочек, точек, крестиков, букв и др.) получались рисунки, напоминающие мозаику. Так печатались графики функций, изображения течений жидкостей и газов, изображения электрических и магнитных полей и т.д. С помощью символьной печати программисты умудрялись получать даже художественные изображения. 

Затем появились специальные устройства для графического вывода на бумагу — графопостроители (плоттеры). С помощью такого устройства на лист бумаги чернильным пером наносятся графические изображения: графики, диаграммы, технические чертежи и прочее.

Но настоящая революция в компьютерной графике произошла с появлением графических дисплеев. На экране графического дисплея стало возможным получать рисунки, чертежи в таком же виде, как на бумаге с помощью карандашей, красок или чертежных инструментов\cite{Vasiliev2005}.

\section{ОСНОВНЫЕ ВИДЫ КОМПЬЮТЕРНОЙ ГРАФИКИ}
Одним из важнейших понятий в компьютерной графике является то, как мы представляем и храним графическую информацию; в первую очередь это относится к изображениям. 

Компьютерная графика подразделяются на двухмерную и трехмерную. Основным объектом двухмерной растровой графики является дискретная плоскость (точнее, ее прямоугольная область), а основным объектом трехмерной растровой графики — трехмерное дискретное пространство. Элементы этого пространства (обычно кубической формы) называются вокселами\cite{Balykina2008}. Трехмерная компьютерная графика широко используется в кино и компьютерных играх.

Несмотря на то, что для работы с компьютерной графикой существует множество классов программного обеспечения, различают всего несколько видов самой компьютерной графики. Обычно выделяют векторную и растровую графику, хотя обособляют еще и фрактальный тип представления изображений. Они отличаются принципами формирования изображения при отображении на экране монитора или при печати на бумаге.

\subsection{ВЕКТОРНАЯ ГРАФИКА}
На ранних стадиях развития компьютерной графики, когда компьютеры были маломощными и обладали очень небольшим объемом оперативной памяти, наиболее распространенным способом представления изображений был векторный.

При векторном способе отображения/хранения все данные представлены как наборы отрезков, дуг и т.п., т.е. линейных примитивов. Такой способ позволяет хранить и отображать чертежи с высокой точностью и требует относительно небольшого объема памяти\cite{Boreskov2017}. При создании картинок применяют формулы и координаты. Например, для создания треугольника необходимо прописать его вершины, цвет заполнения и обводку. При формировании сложных изображений применяют несколько геометрических фигур, которые совмещаются друг с другом. Вместе с тем, такую картинку можно редактировать.

\textbf{ПРЕИМУЩЕСТВА ВЕКТОРНОЙ ГРАФИКИ:}
\begin{enumerate}
    \item Размер, занимаемой описательной частью, не зависит от реальной величины объекта, что позволяет, используя минимальное количество информации, описать сколько угодно раз большой объект файлом минимального размера.
    \item В связи с тем, что информация об объекте хранится в описательной форме, можно бесконечно увеличить графический примитив, например, дугу окружности, и она останется гладкой. С другой стороны, если кривая представлена в виде ломаной линии, увеличение покажет, что она на самом деле не кривая.
    \item Параметры объектов хранятся и могут быть легко изменены. Также это означает что перемещение, масштабирование, вращение, заполнение и т. д. не ухудшат качества рисунка. Более того, обычно указывают размеры в аппаратно"=независимых единицах (англ. device"=independent unit), которые ведут к наилучшей возможной растеризации на растровых устройствах.
    \item  При увеличении или уменьшении объектов толщина линий может быть задана постоянной величиной, независимо от реального контура.
\end{enumerate}

\textbf{НЕДОСТАТКИ ВЕКТОРНОЙ ГРАФИКИ:}
\begin{enumerate}
    \item Не каждый объект может быть легко изображен в векторном виде "--- для подобного оригинальному изображению может потребоваться очень большое количество объектов и их сложности, что негативно влияет на количество памяти, занимаемой изображением, и на время для его отображения (отрисовки).
    \item Перевод векторной графики в растр достаточно прост. Но обратного пути, как правило, нет "--- трассировка растра, при том что требует значительных вычислительных мощностей и времени, не всегда обеспечивает высокое качества векторного рисунка.
\end{enumerate}

\subsection{РАСТРОВАЯ ГРАФИКА}
Растровое изображение "--- изображение, представляющее собой сетку пикселов или точек цветов (обычно прямоугольную) на компьютерном мониторе, бумаге и других отображающих устройствах и материалах.

Растр "--- это матрица ячеек (пикселов). Каждый пиксел может иметь свой цвет. Совокупность пикселов различного цвета образует изображение. В зависимости от расположения пикселов в пространстве различают квадратный, прямоугольный, гексагональный или иные типы растра. 

Для описания расположения пикселов используют разнообразные системы координат. Общим для всех таких систем является то, что координаты пикселов образуют дискретный ряд значений (необязательно целые числа). Часто используется система целых координат — номеров пикселов с (0,0) в левом верхнем уголку. Такую систему мы будем использовать и в дальнейшем, ибо она удобна для рассмотрения алгоритмов графического вывода\cite{Porev2002}.

\textbf{ПРЕИМУЩЕСТВА РАСТРОВОЙ ГРАФИКИ:}
\begin{enumerate}
    \item Растровая графика позволяет создать (воспроизвести) практически любой рисунок, вне зависимости от сложности, в отличие, например, от векторной, где невозможно точно передать эффект перехода от одного цвета к другому без потерь в размере файла.
    \item Распространенность — растровая графика используется сейчас практически везде: от маленьких значков до плакатов. 
    \item Растровое представление изображения естественно для устройств ввода"=вывода графической информации, таких как мониторы (за исключением векторных), матричные и струйные принтеры, цифровые фотоаппараты, сканеры.
    \item Высокая скорость обработки сложных изображений, если не нужно масштабирование.
\end{enumerate}

\textbf{НЕДОСТАТКИ РАСТРОВОЙ ГРАФИКИ:}
\begin{enumerate}
    \item Большой размер файлов с простыми изображениями.
    \item Невозможность идеального масштабирования.
    \item Невозможность вывода на печать на плоттер.
\end{enumerate}

Из"=за этих недостатков для хранения простых рисунков рекомендуют вместо даже сжатой растровой графики использовать векторную графику.

\subsection{ФРАКТАЛЬНАЯ ГРАФИКА}
Во фрактальной графике используется принцип наследования геометрических качеств, которые передаются от фрагмента к фрагменту. Основное свойство фракталов — самоподобие. Любой микроскопический фрагмент фрактала в том или ином отношении воспроизводит его глобальную структуру. В простейшем случае часть фрактала представляет собой просто уменьшенный целый фрактал\cite{Balykina2008}. Этот подход к графике базируется на математических вычислениях (формулах). В связи с отсутствием необходимости детализированного описания мелких элементов отрисовка объекта может осуществляться с помощью нескольких уравнений.

Компьютер в автоматическом режиме отображает полученные результаты. При этом в памяти устройства не нужно хранить никакие данные. Фрактальный метод используется во многих сферах компьютерной графики, науки и искусства, а сами фракталы — в растровой, векторной и 3D графике. Среди программ для создания таких изображений можно отметить: Fractal Explorer, Apophysis, Mandelbulb3D\cite{GeekBrains}.

Роль фракталов в КГ сегодня достаточно велика. Они приходят на помощь, например, когда требуется, с помощью нескольких коэффициентов, задать линии и поверхности очень сложной формы. С точки зрения компьютерной графики, фрактальная геометрия незаменима при генерации облаков, гор, поверхности моря\cite{Peremitina2012}.



\section{ОБЛАСТИ ПРИМЕНЕНИЯ КОМПЬЮТЕРНОЙ ГРАФИКИ}
Компьютерная графика является сравнительно молодой дисциплиной: первые попытки использовать компьютер не только непосредственно для научных расчетов, но и для отображения информации, относятся к началу 1950"=х гг. Далее, с одной стороны, компьютеры становились все быстрее, увеличивался объем оперативной памяти, а с другой "--- развивались средства отображения информации. Появились графические плоттеры (графопостроители), графические дисплеи, причем разрешение последних все время увеличивалось. Все это способствовало бурному росту компьютерной графики. Причем, если в самом начале получаемые изображения носили явно искусственный (или «компьютерный») вид, то в дальнейшем качество получаемых изображений сильно повысилось и стало практически невозможно отличить построенное на компьютере изображение от фотографии. В то же время появилась возможность создавать графику в режиме реального времени, что привело к разработке широкого спектра различных интерактивных приложений, включающих в себя компьютерные игры. В результате компьютерная графика распространилась в сфере кино и рекламы. Большинство сложных спецэффектов оказалось гораздо дешевле и проще сделать целиком на компьютере, более того, многие графические эффекты можно сделать только таким способом. Компьютерная графика играет огромную роль в современном кинематографе, причем не только для реализации различных спецэффектов, но и, например, для нелинейного видеомонтажа\cite{Boreskov2017}.

С появлением достаточно дешевых персональных компьютеров стали активно создаваться различные компьютерные игры, которые также играют важную роль в развитии компьютерной графики. В отличие от кино, где используют мощные компьютеры и кластеры компьютеров, графика для компьютерных игр строилась с помощью довольно слабых в техническом отношении персональных компьютеров. Кроме того, для игр также требовалось построение изображений с заданной частотой кадров, в противном случае игра просто была бы никому не нужна. Результат уступает по качеству тому, что мы видим в кино, но для игр этого бывает достаточно. В последнее время, однако, наблюдается тенденция постоянного повышения и совершенствования качества графики в играх, оно приближается к кинематографическому. 

Из вышесказанного можно сделать вывод, что наиболее яркой и заметной сферой применения компьютерной графики является индустрия развлечений "--- спецэффекты для фильмов, компьютерные мультфильмы и игры. Однако использование компьютерной графики отнюдь не ограничивается этой сферой. Например, она довольно широко применяется для создания привлекающей внимание рекламы "--- как в виде отдельных изображений, так и видеороликов.

Еще одной областью использования компьютерной графики является архитектура, где очень важно иметь возможность увидеть, как будет выглядеть здание до того, как оно будет построено. Аналогичная потребность часто возникает у инженеров "--- определить вид будущего изделия и прямо по ходу работы изменить его характеристики (например, материал). Важную роль компьютерная графика играет и в сфере образования: она дает возможность увидеть сложные (и часто скрытые от глаз) процессы во всех деталях. Нужна она и при создании различных тренажеров, например для обучения пилотов. Сфера применения компьютерных обучающих систем растет с каждым годом, и далеко не последнюю роль в этом играет компьютерная графика. Интересная область применения компьютерной графики "--- визуализация результатов медицинских исследований. Здесь обычно подразумевается построение объемных изображений, полученных в результате томографии или других похожих процедур. При этом крайне важна возможность управлять визуализацией, задавая различные параметры, области отсечения и т.п.\cite{Boreskov2017}

\section{СВЯЗЬ КОМПЬЮТЕРНОЙ ГРАФИКИ С ИСКУССТВЕННЫМ ИНТЕЛЛЕКТОМ}
Искусственный интеллект (ИИ) радикально изменил многие области, и искусство не является исключением. Появление генеративного искусства на основе ИИ, особенно в области создания изображений, открыло новые возможности для творчества и инноваций. В этом эссе рассматриваются недавние достижения в области генеративного искусства на основе ИИ для создания изображений, с акцентом на ключевые модели, такие как Stable Diffusion, Midjourney и DALL"=E.

Глубокая нейронная сеть (ГНС) представляет собой сложную программу, состоящую из большого количества внутренних (скрытых) слоев с настраиваемыми параметрами — весовыми коэффициентами искусственных нейронов, составляющих каждый слой сети. На первый, входной слой сеть принимает вектор признаков, описывающих объект — данные в виде сигналов. На внутренних слоях происходит их обработка: входной вектор умножается на матрицу связей, и сформированный таким образом вектор новых признаков передается в следующий слой. Результат обработки сигнала отправляется на выходной слой сети\cite{PostScience}.

ГНС состоят из взаимосвязанных узлов или <<нейронов>>, которые имитируют функционирование нейронов человеческого мозга. Каждый нейрон принимает на вход данные, обрабатывает их и передает результат на следующий слой. Глубина этих сетей, то есть количество скрытых слоев, позволяет им изучать и моделировать сложные закономерности в данных.

В контексте генеративного искусства на основе ИИ, ГНС используются для понимания сложных связей между входными запросами (обычно текстом) и выходными данными (изображением). Они обучаются на больших наборах данных пар текст"=изображение, учась генерировать изображения, соответствующие данным текстовым описаниям.

Как работает ИИ для преобразования текста в изображение?
Генеративные модели ИИ текст"=в"=изображение "--- это тип моделей машинного обучения, которые принимают на вход описание на естественном языке и создают изображение, соответствующее этому описанию. Эти модели обычно объединяют языковую модель, которая преобразует входной текст в скрытое представление, и генеративную модель изображения, которая создает изображение, основанное на этом представлении.

Процесс обычно включает следующие этапы.

\begin{enumerate}
    \item Кодирование текста: Входной текст преобразуется в скрытое представление с помощью языковой модели. Это может быть рекуррентная нейронная сеть, такая как сеть долгой краткосрочной памяти (LSTM), или модель трансформера.
    \item Генерация изображения: Затем генеративная модель изображения создает изображение, основанное на скрытом представлении. Для этого шага обычно используются условные генеративные состязательные сети (GAN) и модели диффузии.
    \item Масштабирование: Вместо того, чтобы напрямую обучать модель выводить изображение высокого разрешения, основанное на текстовом встраивании, популярной техникой является обучение модели генерировать изображения низкого разрешения и использовать одну или несколько вспомогательных моделей глубокого обучения для увеличения его разрешения, добавляя более тонкие детали.
\end{enumerate}

\subsection{НЕДАВНИЕ ДОСТИЖЕНИЯ}
\subsubsection{Stable Diffusion}
Stable Diffusion, разработанная Stability AI в 2022 году, "--- это модель текст"=в"=изображение, которая может генерировать детализированные изображения на основе текстовых описаний. Она основана на архитектуре скрытой диффузии, которая состоит из вариационного автоэнкодера (VAE), U"=Net и необязательного текстового кодировщика. Она использует процесс удаления шума, при котором случайный шум итеративно применяется к сжатому скрытому представлению изображения, руководствуясь текстовым запросом. Полученные изображения получаются путем удаления шума из скрытого представления и декодирования его обратно в пиксельное пространство\cite{VegaIT}.

\subsubsection{DALL"=E И DALL"=E 2}
DALL"=E и DALL"=E 2 "--- это модели глубокого обучения, разработанные OpenAI. DALL"=E, представленный в январе 2021 года, использует модифицированную версию GPT"=3 для генерации изображений. 

DALL"=E 2, представленный в апреле 2022 года, "--- это улучшенная версия, способная генерировать более реалистичные изображения с более высоким разрешением и комбинировать концепции, атрибуты и стили. DALL"=E 2 использует модель диффузии, условно основанную на встраиваниях изображений CLIP\cite{TechTarget}.

\subsubsection{Midjourney}
Midjourney "--- это генеративная программа искусственного интеллекта, разработанная компанией Midjourney, Inc. Подобно DALL"=E от OpenAI и Stable Diffusion, Midjourney генерирует изображения из описаний на естественном языке, называемых <<подсказками>>. Инструмент вошел в открытую бета"=версию 12 июля 2022 года и доступен через бота Discord на официальном сервере или путем приглашения бота на сторонний сервер.

Под руководством Дэвида Хольца, сооснователя Leap Motion, Midjourney выпустила несколько версий своего алгоритма, включая V1 в феврале 2022 года, V2 в апреле 2022 года, V3 в июле 2022 года, V4 alpha в ноябре 2022 года и V5 alpha в марте 2023 года. Модель 5.1, выпущенная в мае 2023 года, применяет собственную стилизацию к изображениям, в то время как модель 5.1 RAW улучшает совместимость с более буквальными подсказками\cite{BBC}.


\conclusion
Благодаря разработкам как в аппаратном обеспечении графических процессоров, так и в программном обеспечении механизмов рендеринга разработки в области компьютерной графики продолжают расширять границы как точности, так и скорости создания изображений, генерируемых компьютером\cite{Cornell}.

Достижения в области генеративного искусства на основе ИИ для создания изображений открыли новые пути для творчества и инноваций. Модели, такие как Stable Diffusion, DALL"=E и Midjourney, продемонстрировали потенциал ИИ в генерации высококачественных, творческих изображений из текстовых подсказок. По мере того как эти модели продолжают эволюционировать, мы можем ожидать еще более впечатляющих возможностей и приложений в будущем.
% Библиографический список, составленный вручную, без использования BibTeX
%
%\begin{thebibliography}{99}
%    \bibitem{Ione} Источник 1.
%    \bibitem{Itwo} Источник 2
%\end{thebibliography}

% Отобразить все источники. Даже те, на которые нет ссылок.
% \nocite{*}

% Меняем inputencoding на лету, чтобы работать с библиографией в кодировке
% `cp1251', в то время как остальной документ находится в кодировке `utf8'
% Credit: Никита Рыданов
\inputencoding{cp1251}
\bibliographystyle{gost780uv}
\bibliography{thesis}
\inputencoding{utf8}

% При использовании biblatex вместо bibtex
% \printbibliography

% Окончание основного документа и начало приложений Каждая последующая секция
% документа будет являться приложением
\appendix

\end{document}
