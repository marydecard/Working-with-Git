Нашла Маша гриб, сорвала его и понеслана рынок. На рынке Машу ударили по голове, да еще обещали ударить ее по ногам. Испугалась Маша и побежала прочь.

Прибежала Маша в кооператив и хотела там за кассу спрятаться. А заведующий увидел Машу и говорит:

"--~ Что это у тебя в руках?

А Маша говорит:

"--~ Гриб.

Заведующий говорит:

"--~ Ишь какая бойкая! Хочешь, я тебя на место устрою?

Маша говорит:

"--~ А не устроишь.

Заведующий говорит:

"--~ А вот устрою!"--~ и устроил Машу кассу вертеть.

Маша вертела, вертела кассу и вдруг умерла. Пришла милиция, составила протокол и
велела заведующему заплатить штраф"--~ 15 рублей.

Заведующий говорит:
 
"--~ За что же штраф?
 
А милиция говорит:

"--~ За убийство.
 
Заведующий испугался, заплатил поскорее штраф и говорит:
 
"--~ Унесите только поскорее эту мертвую кассиршу.
 
А продавец из фруктового отдела говорит:
 
"--~ Нет, это неправда, она была не кассирша. Она только ручку в кассе вертела. А кассирша вон сидит.
 
Милиция говорит:
 
"--~ Нам все равно: сказано унести кассиршу, мы ее и унесем.
 
Стала милиция к кассирше подходить.
 
Кассирша легла на пол за кассу и говорит:
 
"--~ Не пойду.
 
Милиция говорит:
 
"--~ Почему же ты, дура, не пойдешь?

Кассирша говорит:

"--~ Вы меня живой похороните.
 
Милиция стала кассиршу с пола поднимать, но никак поднять не может, потому что кассирша очень полная.
 
"--~ Да вы ее за ноги,"--~ говорит продавец из фруктового отдела.
 
"--~ Нет,"--~ говорит заведующий,"--~ эта кассирша мне вместо жены служит. А потому прошу вас, не оголяйте ее снизу.
 
Кассирша говорит:
 
"--~ Вы слышите? Не смейте меня снизу оголять.
 
Милиция взяла кассиршу под мышки и волоком выперла ее из кооператива.
 
Заведующий велел продавцам прибрать магазин и начать торговлю.
 
"--~ А что мы будем делать с этой покойницей?"--~ говорит продавец из фруктового отдела, показывая на Машу.

"--~ Батюшки,"--~ говорит заведующий,"--~ да ведь мы все перепутали! Ну, действительно, что с покойницей делать?
 
"--~ А кто за кассой сидеть будет?"--~ спрашивает продавец.
 
Заведующий за голову руками схватился. Раскидал коленом яблоки по прилавку и говорит:
 
"--~ Безобразие получилось!
 
"--~ Безобразие,"--~ говорит хором продавцы.
 
Вдруг заведующий почесал усы и говорит:
 
"--~ Хе"=хе! Не так"=то легко меня в тупик поставить! Посадим покойницу за кассу, может, публика и не разберет, кто за кассой сидит.
 
Посадили покойницу за кассу, в зубы ей папироску вставили, чтобы она на живую больше походила, а в руки для правдоподобности дали ей гриб держать.
 
Сидит покойница за кассой, как живая, только цвет лица очень зеленый, и один глаз открыт, а другой совершенно закрыт.
 
"--~ Ничего,"--~ говорит заведующий,"--~ сойдет.
 
А публика уже в двери стучит, волнуется. Почему кооператив не открывают? Особенно одна хозяйка в шелковом манто раскричалась: трясет кошелкой и каблуком уже в дверную ручку нацелилась. А за хозяйкой какая"=то старушка с наволочкой на голове, кричит, ругается и заведующего кооперативом называет сквалыжником.
 
Заведующий открыл двери и впустил публику. Публика побежала сразу в мясной отдел, а потом туда, где продается сахар и перец. А старушка прямо в рыбный отдел пошла, но по дороге взгянула на кассиршу и остановилась.
 
"--~ Господи,"--~ говорит,"--~ с нами крестная сила!
 
А хозяйка в шелковом манто уже во всех отделах побывала и несется прямо к кассе. Но только на кассиршу взгянула, сразу остановилась, стоит молча и смотрит. А продавцы тоже молчат и смотрят на заведующего. А заведующий из"=за прилавка выглядывает и ждет, что дальше будет.
 
Хозяйка в шелковом манто повернулась к продавцам и говорит:

"--~ Это кто у вас за кассой сидит?
 
А продавцы молчат, потому что не знают, что ответить.
 
Заведующий тоже молчит.
 
А тут народ со всех сторон сбегается. Уже на улице толпа. Появились дворники. Раздались свистки. Одним словом, настоящий скандал.
 
Толпа готова была хоть до самого вечера стоять около кооператива, но кто"=то сказал, что в Озерном переулке из окна старухи вываливаются. Тогда толпа возле кооператива поредела, потому что многие перешли в Озерный переулок.

\begin{flushright}
    31 августа 1936 года.
\end{flushright}