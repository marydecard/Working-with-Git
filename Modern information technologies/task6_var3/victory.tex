Мышину сказали: \textit{<<Эй, Мышин, вставай!>>}

Мышин сказал: \textit{<<Не встану>>},"--- и продолжал лежать на полу.

Тогда к Мышину подошел Кулыгин и сказал:

\textit{<<Если ты, Мышин, не встанешь, я тебя заставлю встать>>}. \textit{<<Нет>>},"--- сказал Мышин, продолжая лежать на полу. К Мышину подошла Селезнева и сказала: \textit{<<Вы, Мышин, вечно валяетесь на полу в коридоре и мешаете нам ходить взад и вперед>>}.

"--~ \textit{Мешал и буду мешать},"--- сказал Мышин.

"--~ \textit{Ну знаете},"--- сказал Коршунов, но его перебил Кулыгин и сказал:

"--~ \textit{Да чего тут долго разговаривать! Звоните в милицию}.

Позвонили в милицию и вызвали милиционера.

Через полчаса пришел милиционер с дворником.

"--~ \textit{Чего у вас тут?}"--- спросил милиционер.

"--~ \textit{Полюбуйтесь},"--- сказал Коршунов, но его перебил Кулыгин и сказал:

"--~ \textit{Вот. Этот гражданин все время лежит тут на полу и мешает нам ходить по коридору. Мы его и так и эдак...}

Но тут Кулыгина перебила Селезнева и сказала:

"--~ \textit{Мы его просили уйти, а он не уходит}.

"--~ \textit{Да},"--- сказал Коршунов.

Милиционер подошел к Мышину.

"--~ \textit{Вы, гражданин, зачем тут лежите?}"--- сказал милиционер.

"--~ \textit{Отдыхаю},"--- сказал Мышин.

"--~ \textit{Здесь, гражданин, отдыхать не годится},"--- сказал милиционер. "--~ \textit{Вы где, гражданин, живете?}

"--~ \textit{Тут},"--- сказал Мышин.

"--~ \textit{Где ваша комната?}"--- спросил милиционер.

"--~ \textit{Он прописан в нашей квартире, а комнаты не имеет},"--- сказал Кулыгин.

"--~ \textit{Обождите, гражданин},"--- сказал милиционер,"--- \textit{я сейчас с ним говорю. Гражданин, где вы спите?}

"--~ \textit{Тут},"--- сказал Мышин.

"--~ \textit{Позвольте},"--- сказал Коршунов, но его перебил Кулыгин и сказал:

"--~ \textit{Он даже кровати не имеет и валяется на голом полу}.

"--~ \textit{Они давно на него жалуются},"--- сказал дворник.

"--~ \textit{Совершенно невозможно ходить по коридору},"--- сказала Селезнева. "--~ \textit{Я не могу вечно шагать через мужчину. А он нарочно ноги вытянет, да еще руки вытянет, да еще на спину ляжет и глядит. Я с работы усталая прихожу, мне отдых нужен}.

"--~ \textit{Присовокупляю},"--- сказал Коршунов, но его перебил Кулыгин и сказал:

"--~ \textit{Он и ночью здесь лежит. Об него в темноте все спотыкаются. Я через него одеяло свое разорвал}.

Селезнева сказала:

"--~ \textit{У него вечно из кармана какие"=то гвозди вываливаются. Невозможно по коридору босой ходить, того и гляди ногу напорешь}.

"--~ \textit{Они давеча хотели его керосином пожечь},"--- сказал дворник.

"--~ \textit{Мы его керосином облили},"--- сказал Коршунов, но его перебил Кулыгин и сказал:

"--~ \textit{Мы его только для страха облили, а поджечь и не собирались}.

"--~ \textit{Да я бы не позволила в своем присутствии живого человека жечь},"--- сказала Селезнева.

"--~ \textit{А почему этот гражданин в коридоре лежит?}"--- спросил вдруг милиционер.

"--~ \textit{Здрасьте"=пожалуйста!}"--- сказал Коршунов, но Кулыгин его перебил и сказал:

"--~ \textit{А потому что у него нет другой жилплощади: вот в этой комнате я живу, в той"--- вот они, в этой"--- вот он, а уж Мышин в коридоре живет}.

"--~ \textit{Это не годится},"--- сказал милиционер. "--~ \textit{Надо, чтобы все на своей жилплощади лежали}.

"--~ \textit{А у него нет другой жилплощади, как в коридоре},"--- сказал Кулыгин.

"--~ \textit{Вот именно},"--- сказал Коршунов.

"--~ \textit{Вот он вечно тут и лежит},"--- сказала Селезнева.

"--~ \textit{Это не годится},"--- сказал милиционер и ушел вместе с дворником.

Коршунов подскочил к Мышину.

"--~ \textit{Что?}"--- закричал он. "--~ \textit{Как вам это по вкусу пришлось?}

"--~ \textit{Подождите},"--- сказал Кулыгин и, подойдя к Мышину, сказал. "--~ \textit{Слышал, чего говорил милиционер? Вставай с полу}.

"--~ \textit{Не встану},"--- сказал Мышин, продолжая лежать на полу.

"--~ \textit{Он теперь нарочно и дальше будет вечно тут лежать},"--- сказала Селезнева.

"--~ \textit{Определенно},"--- сказал с раздражением Кулыгин.

И Коршунов сказал:

"--~ \textit{Я в этом не сомневаюсь. Parfaifemenf!}

\begin{flushright}
    <\dots>
\end{flushright}