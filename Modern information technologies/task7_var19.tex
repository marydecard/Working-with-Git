\documentclass{article}

\usepackage[T2A]{fontenc}
\usepackage[utf8]{inputenc}
\usepackage{amsthm}
\usepackage{amsmath}
\usepackage{amssymb}
\usepackage{amsfonts}
\usepackage{mathrsfs}
\usepackage[12pt]{extsizes}
\usepackage{fancyvrb}
\usepackage{indentfirst}
\usepackage[left=2cm, right=2cm, top=2cm, bottom=2cm, headsep=0.2cm, footskip=0.6cm, bindingoffset=0cm]{geometry}
\usepackage[english,russian]{babel}

\begin{document}

\section*{Вариант 19}
Упорядоченный группоид $(A, \cdot, \leqslant)$ тогда и только тогда принадлежит классу \textbf{R}$\{ \ast, \subset\}$, когда для всякой $n$"=диады $\omega = (\alpha, \beta)$ и любых термов $p_1, \dots, p_n;~\tilde{p_0}, \dots, \tilde{p_n}$ и $q_{i,j}~(i, j = 0, \dots l_n)$ таких, что для $G_{k-1}^{(\alpha(k), \beta(k))} \prec G(p_k)~(k = 1, \dots n)$, $G_{\omega}^{(r,r+1)} \prec G(\tilde{p_t})~(t = 0, \dots n;~r = 0$ для $t = 0$ и $r = th + 1$ для $t = 1, \dots n)$ и $G_{\omega}^{(i,j)} \prec G(q_{i,j})~(i, j = 0, \dots l_n)$, выполняются аксиомы:

\begin{equation*}
    \left(\bigwedge\limits_{k=0}^{n} x_{3k+1} \neq \textbf{0} \wedge \bigwedge\limits_{k=1}^{n} p_k \leqslant x_{3k-1}x_{3k} \right) \rightarrow x_{3t+1} \leqslant \tilde{p_t},
\end{equation*}

\begin{equation*}
    \left( \bigwedge\limits_{k=0}^{n} x_{3k+1} \neq \textbf{0} \wedge \bigwedge\limits_{k=1}^{n} p_k \leqslant x_{3k-1}x_{3k} \right) \rightarrow q_{i,j} \neq \textbf{0},
\end{equation*}
где \textbf{0} "--- нулевой элемент упорядоченного группоида $(A, \cdot, \leqslant)$.

\end{document}

